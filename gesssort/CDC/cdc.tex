\documentclass[a4paper, 11pt]{report}
\usepackage[top=2cm, bottom=2cm, left=2cm, right=2cm]{geometry}
\usepackage[utf8]{inputenc}
\usepackage[T1]{fontenc}
\usepackage[french]{babel}
\renewcommand{\thesection}{\arabic{section}}
\usepackage{hyperref}
\hypersetup{pdfborder=0 0 0}
\newcommand{\ges}{Gessort }
\author{Paimblanc Cédric}
\date{\today}
\title{Cahier des charges d'une Application de Gestion de sort}
\begin{document}
	\maketitle
	\tableofcontents
	\newpage
	\section{Présentation de l'application \ges}
	\label{sec:mise_en_contexte_de_l_application}
		L'application \ges est une application qui va permettre à l'utilisateur de gérer la liste de ses
		sorts, les trier par niveau, par nom, ou par domaine. Il pourra effectuer une recherche pour
		chercher un sort en particulier et en consulter la description. Il aura aussi la possibilité
		d'en ajouter, d'en supprimer de les modifier.
	\section{Les besoins et les contraintes}	
	\label{sec:Les besoins et les contraintes}
		\subsection{Les besoins}
			\paragraph{Présentation} Lors d'une séance de jeu de rôle, les lanceurs de sort possèdent 
			une liste de sort qui s'allonge avec la montée des niveaux, et parfois retrouver la description 
			d'un sort en particulier peut être fastidieux, et peux faire perdre la 
			la fluidité de la séance en cours et c'est d'autant plus vrai 
			si on utilise plusieurs livres.\\
			Actuellement un joueur incarnant  un lanceur de sort doit soit écrire toute la description 
			complète de tous ses sorts  et finalement avoir des dizaines de  pages, ou écrire une 
			description plus brève au risque que certaines informations ne soit pas notés, et qui 
			nécessite finalement une recherche difficile et longue pour plus de précisions. 
			\paragraph{La réponse} L'application a pour but de concilier les deux, c'est à dire, 
			avoir une liste avec une description brève qui permet en un seul coup d'œil de trouver 
			un sort qui peut répondre à un besoin donnée, mais qui permet lorsque l'on sélectionne le dit sort d'avoir une 
			description détaillé. De plus comme il a été dit dans l'introduction le joueur devra avoir
			la possibilité de faire une recherche pour trouver un sort précis.  
		\subsection{Les contraintes}
			Comme toute applications qui se respectent, on cherche une ergonomie le plus aboutie possible.
			L'application doit permettre une navigation fluide et intuitive, on ne veut pas que ce soit 
			surcharger, et elle doit être utilisable si possible sur la plupart des smartphones et des 
			tablettes android. 
			\subsection{Version 1}
				En toute premier lieu, l'application doit pouvoir afficher la liste complète des sorts
				que l'utilisateur aura ajouté, chaque sort doit pouvoir afficher un description complète
				dans une autre page lorsqu'il est sélectionné. L'utilisateur doit pouvoir supprimer ses 
				sorts, les mettre à jour pour une raison quelconque.  
			\subsection{Version 2}
				La première amélioration à apporter sera d'afficher les descriptions brèves sous le nom des 
				sorts selon les possibilités techniques, une barre de recherche devra être ajoutée afin 
				que l'utilisateur puisse chercher un sort spécifique sans avoir à parcourir toute sa liste,
				une persistance des recherches facilitera des recherches futur.
			\subsection{Version 3} 
				La troisième version de l'application devra permettre d'afficher les sorts triés.
				On veut pouvoir les afficher par lettre, par niveaux, ou par domaine. Une 
				internationalisation pourra être envisagée.

	%\section{Les maquettes}
	%\label{sec:les_maquettes}
	
		

\end{document}
